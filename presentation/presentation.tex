\documentclass{beamer}
 
\usepackage[utf8]{inputenc}
 
 
%Information to be included in the title page:
\title{What i learned during TechDoc}
\author{Dennis Houterman}
\date{2017}
 
 
 
\begin{document}
 
\frame{\titlepage}
 
\begin{frame}
\frametitle{The inner workings of LaTeX}
Separates text from layout and added elements and make things happen in your document with commands
\end{frame}

\begin{frame}
\frametitle{Working with LaTeX editors}
Personally, i like both Texmaker and ShareLaTeX. if you install Texmaker though, be sure to install the latex compiler too.
\end{frame}

\begin{frame}
\frametitle{Gitting Guud}
with git it is possible to work on documents with multiple people at a time. not only that however, it is also possible to "branch" off to have the project all to yourself, and merge it with the original project if you finished a feature like a paragraph of text or a new part of your program in code 
\end{frame}

\begin{frame}
\frametitle{More then a thousand words}
adding vector images to your LaTeX document is easy. use the {graphicx} package and add an pdf image by using: includegraphics[height=5cm]{drawing.pdf}
\end{frame}
 
\begin{frame}
\maketitle{using fonts}
fonts are cool i guess...
\end{frame} 

\begin{frame}
\frametitle{The wonderfull world of document formats}
books you buy online are most of the time in pdf, epub or similar formats. these consist of the plain text and the formatting information. in epub for example, this is html for the text and css for the layout of the text and elements used in the book.
\end{frame} 

\begin{frame}
\frametitle{conclusions}
LaTeX is very versatile and easy to work with if you dont mind looking up commands when you start making such documents
git is a great way to manage projects if you want to work on the same files as your project partners
\end{frame}
 
\end{document}